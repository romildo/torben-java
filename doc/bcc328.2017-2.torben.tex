\documentclass[a4paper,10pt,brazil]{article}

\usepackage[left=25mm,top=20mm]{geometry}
\usepackage[utf8]{inputenc}
\usepackage[T1]{fontenc}
\usepackage{babel}
\usepackage{hyperref}
\usepackage{lmodern}
\usepackage{indentfirst}
\usepackage{tabularx}
\usepackage{graphics}
\usepackage{syntax}
\usepackage{pygmentex}
\usepackage{jrmmisc}
\usepackage{task}

\setpygmented{lang=text,tabsize=4}
\efboxsetup{hidealllines,backgroundcolor=yellow!30}

\mdfsetup{
  % backgroundcolor=green!10,
  skipabove=2pt,
  skipbelow=2pt,
  innerleftmargin=2pt,
  innerrightmargin=2pt,
  innertopmargin=.25\baselineskip,
  innerbottommargin=.25\baselineskip,
}

% syntax configuration
\renewcommand{\syntleft}{\normalfont\slshape\hspace{0.25em}}
\renewcommand{\syntright}{\hspace{0.25em}}
\renewcommand{\ulitleft}{\normalfont\ttfamily\bfseries\frenchspacing\color{red}\hspace{0.25em}}
\renewcommand{\ulitright}{\hspace{0.25em}}



\newcommand{\lang}{\textsl{Torben}}

\begin{document}

\begin{center}
  \textbf{Construção de Compiladores I [BCC328]}
\end{center}
\begin{tcolorbox}[center title,colbacktitle=yellow!20,coltitle=blue,fonttitle=\scshape,title=\textbf{\textsf{Atividade Prática}}]
  \begin{center}
    \textbf{\Large Análise Léxica da Linguagem \lang{}}
  \end{center}
\end{tcolorbox}
\begin{center}
  Departamento de Computação\par
  Universidade Federal de Ouro Preto\par
  Prof. José Romildo Malaquias\par
  \today
\end{center}

\begin{abstract}
  Nesta atividade vamos implementar os analisadores léxico e sintático
  para uma pequena linguagem de programação.
\end{abstract}

\tableofcontents

\section{A linguagem \lang{}}

Torben Ægidius Mogensen apresenta no capítulo 4 de seu livro
\textsl{Introduction to Compiler Design} uma pequena linguagem de
programação para fins didáticos.

Nas atividades propostas neste roteiro vamos considerar uma linguagem,
que chamaremos de \lang{}, baseada nessa linguagem, porém estendida para
incluir:
\begin{itemize}
  \item comentários de linha,
  \item comentários de bloco,
  \item literais booleanos,
  \item o tipo \texttt{string},
  \item literais strings,
  \item operadores aritméticos: \texttt{-} (simétrico e subtração),
  \texttt{*} (multiplicação), \texttt{/} (divisão), e \texttt{\%}
  (resto da divisão inteira),
  \item operadores relacionais: \texttt{<>} (diferente), \texttt{>}
  (maior que), \texttt{>=} (maiour ou igual a), \texttt{<} (menor que),
  e \texttt{<=} (menor ou igual a),
  \item operadores lógicos: \texttt{\&\&} (e lógico), e \texttt{||} (ou
  lógico).
  \item atribuição,
  \item expressão de repetição, e
  \item expressão sequência.
\end{itemize}

Apresentamos a seguir uma gramática livre de contexto (com comentários)
para \lang{}, que define a sintaxe de todas as construções da linguagem.

\begin{synshorts}
  \begin{center}
    \small
    \begin{tabular}{r@{$\;\rightarrow\;$}l>{\bfseries}r}
      <Program>     & <Funs>                               & programa                \\[.9em]
      <Funs>        & <Fun>                                &                         \\
      <Funs>        & <Fun> <Funs>                         &                         \\[.9em]
      <Fun>         & <TypeId> "(" <TypeIds> ")" "=" <Exp> & declaração de função    \\
      <TypeId>      & "bool" "id"                          & tipo booleano           \\
      <TypeId>      & "int" "id"                           & tipo inteiro            \\
      <TypeId>      & "string" "id"                        & tipo string             \\
      <TypeIds>     &                                      & lista de parâmetros     \\
      <TypeIds>     & <TypeId> <TypeIdsRest>               &                         \\
      <TypeIdsRest> &                                      &                         \\
      <TypeIdsRest> & "," <TypeId> <TypeIdsRest>           &                         \\[.9em]
      <Exp>         & "litbool"                            & literais                \\
      <Exp>         & "litint"                             &                         \\
      <Exp>         & "litstring"                          &                         \\
      <Exp>         & "id"                                 & variável                \\
      <Exp>         & "id" ":=" <Exp>                      & atribuição              \\
      <Exp>         & <Exp> "+" <Exp>                      & operações aritméticas   \\
      <Exp>         & <Exp> "-" <Exp>                      &                         \\
      <Exp>         & <Exp> "*" <Exp>                      &                         \\
      <Exp>         & <Exp> "/" <Exp>                      &                         \\
      <Exp>         & <Exp> "\%" <Exp>                     & resto da divisão        \\
      <Exp>         & "-" <Exp>                            &                         \\
      <Exp>         & <Exp> "=" <Exp>                      & operações relacionais   \\
      <Exp>         & <Exp> "<>" <Exp>                     &                         \\
      <Exp>         & <Exp> ">" <Exp>                      &                         \\
      <Exp>         & <Exp> ">=" <Exp>                     &                         \\
      <Exp>         & <Exp> "<" <Exp>                      &                         \\
      <Exp>         & <Exp> "<=" <Exp>                     &                         \\
      <Exp>         & <Exp> "\&\&" <Exp>                   & operações lógicas       \\
      <Exp>         & <Exp> "||" <Exp>                     &                         \\
      <Exp>         & "id" "(" <Exps> ")"                  & chamada de função       \\
      <Exp>         & "if" <Exp> "then" <Exp> "else" <Exp> & expressão condicional   \\
      <Exp>         & "while" <Exp> "do" <Exp>             & expressão de repetição  \\
      <Exp>         & "let" "id" "=" <Exp> "in" <Exp>      & expressão de declaração \\
      <Exp>         & "(" <Exps> ")"                       & expressão sequência     \\[.9em]
      <Exps>        &                                      &                         \\
      <Exps>        & <Exp> <ExpsRest>                     &                         \\
      <ExpsRest>    &                                      &                         \\
      <ExpsRest>    & "," <Exp> <ExpsRest>                 &                         \\
    \end{tabular}
  \end{center}
\end{synshorts}

A precedência relativa e a associatividade dos operadores é indicada
pela tabela a seguir, em ordem decrescente de precedência.
\begin{center}
  \begin{tabular}{|l|l|}\hline
    \textbf{operadores}                                                       & \textbf{associatividade} \\\hline
    \texttt{-} (unário)                                                       &                          \\\hline
    % \texttt{\textasciicircum}                                                 & direita                  \\\hline
    \texttt{*}, \texttt{/}, \texttt{\%}                                       & esquerda                 \\\hline
    \texttt{+}, \texttt{-} (binário)                                          & esquerda                 \\\hline
    \texttt{=}, \texttt{<>}, \texttt{>}, \texttt{>=}, \texttt{<}, \texttt{<=} &                          \\\hline
    \texttt{\&\&}                                                             & esquerda                 \\\hline
    \texttt{||}                                                               & esquerda                 \\\hline
    \texttt{:=}                                                               & direita                  \\\hline
    \texttt{then}, \texttt{else}, \texttt{do}, \texttt{in}                    & direita                  \\\hline
  \end{tabular}
\end{center}


Observe que um programa em \lang{} é uma sequência de declarações de
funções.

Um programa deve definir uma função sem argumentos chamada \texttt{main}
que resulta em um inteiro. A execução do programa inicia-se pela chamada
desta função \texttt{main}.


\section{Aspectos léxicos}

\textbf{Comentários de linha} em \lang{} começam com o caracter
\texttt{\#} e se estendem até o final da linha. \textbf{Comentários de
  bloco} são delimitados pelas sequências de caracteres \texttt{\{\#}
e \texttt{\#\}} e podem ser aninhados.

Ocorrências de \textbf{caracteres brancos} (espaço, tabulação horizontal
e nova linha) e comentários entre os símbolos léxicos são ignoradas,
servindo apenas para separar símbolos léxicos.

Os \textbf{literais inteiros} são formados por uma sequência de um ou
mais dígitos decimais.

% Os \textbf{literais reais} são formados por uma sequência de um ou mais
% dígitos decimais seguida do símbolo \texttt{.}, seguido de uma sequência
% de um ou mais dígitos decimais. Uma e somente uma das duas sequências de
% dígitos é opcional.

Os \textbf{literais booleanos} são \texttt{true} (verdadeiro) e
\texttt{false} (falso).

% Os \textbf{literais caracteres} são formados por um caracter gráfico ou
% uma sequência de escape delimitada pelo símbolo \texttt{'}. As únicas
% sequências de escape válidas são as indicadas na tabela a seguir.
% \begin{center}
%   \begin{tabular}{|l|l|}\hline
%     \textbf{sequência de escape}                      & \textbf{descrição}                                                              \\\hline
%     \texttt{\textbackslash \textbackslash}            & \texttt{\textbackslash}                                                         \\\hline
%     \texttt{\textbackslash '}                         & \texttt{'}                                                                      \\\hline
%     \texttt{\textbackslash t}                         & tabuação horizontal                                                             \\\hline
%     \texttt{\textbackslash n}                         & nova linha                                                                      \\\hline
%     \texttt{\textbackslash r}                         & retorno de carro                                                                \\\hline
%     \texttt{\textbackslash b}                         & backspace                                                                       \\\hline
%     \texttt{\textbackslash \textasciicircum \emph{c}} & caracter de controle \emph{c}, sendo \emph{c} uma letra maiúscula, \texttt{@}, \texttt{[}, \texttt{\textbackslash}, \texttt{]}, \texttt{\textasciicircum}, \texttt{\_} ou \texttt{?} \\\hline
%     \texttt{\textbackslash \emph{ddd}}                & caracter de código \emph{ddd}, sendo \emph{d} qualquer dígito decimal           \\\hline
%   \end{tabular}
% \end{center}

Os \textbf{literais string} são formados por uma sequência de caracteres
gráficos delimitada por aspas (\texttt{"}). Na sequência de caracteres o
caracter \texttt{\textbackslash} é especial e inicia uma sequência de
escape. As únicas sequências de escape válidas são indicados na tabela a
seguir.
\begin{center}
  \begin{tabular}{|l|l|}\hline
    \textbf{sequência de escape}                      & \textbf{descrição}                                                              \\\hline
    \texttt{\textbackslash \textbackslash}            & \texttt{\textbackslash}                                                         \\\hline
    \texttt{\textbackslash "}                         & \texttt{"}                                                                      \\\hline
    \texttt{\textbackslash t}                         & tabuação horizontal                                                             \\\hline
    \texttt{\textbackslash n}                         & nova linha                                                                      \\\hline
    \texttt{\textbackslash r}                         & retorno de carro                                                                \\\hline
    \texttt{\textbackslash f}                         & avanço de formulário                                                            \\\hline
    \texttt{\textbackslash b}                         & backspace                                                                       \\\hline
    % \texttt{\textbackslash \textasciicircum \emph{c}} & caracter de controle \emph{c}, sendo \emph{c} uma letra maiúscula, \texttt{@}, \texttt{[}, \texttt{\textbackslash}, \texttt{]}, \texttt{\textasciicircum}, \texttt{\_} ou \texttt{?} \\\hline
    \texttt{\textbackslash \emph{ddd}}                & caracter de código \emph{ddd}, sendo \emph{d} qualquer dígito decimal           \\\hline
  \end{tabular}
\end{center}

% Os primeiros 32 caracteres na tabela ASCII são códigos de controle não
% imprimíveis e são usados para controlar periféricos, tais como
% impressoras. A tabela~\ref{tab:control} lista todos os caracteres de
% controle ASCII.

% \begin{table}[H]
%   \centering
%   \begin{tabular}{llllll} \hline
%     Decimal & Hexadecimal & Abbreviation & Caret notation & Escape code & Name                         \\ \hline
%     0       & 00          & NUL          & \verb|^@|      & \verb|\0|   & Null character               \\
%     1       & 01          & SOH          & \verb|^A|      &             & Start of Header              \\
%     2       & 02          & STX          & \verb|^B|      &             & Start of Text                \\
%     3       & 03          & ETX          & \verb|^C|      &             & End of Text                  \\
%     4       & 04          & EOT          & \verb|^D|      &             & End of Transmission          \\
%     5       & 05          & ENQ          & \verb|^E|      &             & Enquiry                      \\
%     6       & 06          & ACK          & \verb|^F|      &             & Acknowledgment               \\
%     7       & 07          & BEL          & \verb|^G|      & \verb|\a|   & Bell                         \\
%     8       & 08          & BS           & \verb|^H|      & \verb|\b|   & Backspace                    \\
%     9       & 09          & HT           & \verb|^I|      & \verb|\t|   & Horizontal Tab               \\
%     10      & 0A          & LF           & \verb|^J|      & \verb|\n|   & Line feed                    \\
%     11      & 0B          & VT           & \verb|^K|      & \verb|\v|   & Vertical Tab                 \\
%     12      & 0C          & FF           & \verb|^L|      & \verb|\f|   & Form feed                    \\
%     13      & 0D          & CR           & \verb|^M|      & \verb|\r|   & Carriage return              \\
%     14      & 0E          & SO           & \verb|^N|      &             & Shift Out                    \\
%     15      & 0F          & SI           & \verb|^O|      &             & Shift In                     \\
%     16      & 10          & DLE          & \verb|^P|      &             & Data Link Escape             \\
%     17      & 11          & DC1          & \verb|^Q|      &             & Device Control 1 (oft. XON)  \\
%     18      & 12          & DC2          & \verb|^R|      &             & Device Control 2             \\
%     19      & 13          & DC3          & \verb|^S|      &             & Device Control 3 (oft. XOFF) \\
%     20      & 14          & DC4          & \verb|^T|      &             & Device Control 4             \\
%     21      & 15          & NAK          & \verb|^U|      &             & Negative Acknowledgment      \\
%     22      & 16          & SYN          & \verb|^V|      &             & Synchronous idle             \\
%     23      & 17          & ETB          & \verb|^W|      &             & End of Transmission Block    \\
%     24      & 18          & CAN          & \verb|^X|      &             & Cancel                       \\
%     25      & 19          & EM           & \verb|^Y|      &             & End of Medium                \\
%     26      & 1A          & SUB          & \verb|^Z|      &             & Substitute                   \\
%     27      & 1B          & ESC          & \verb|^[|      & \verb|\e|   & Escape                       \\
%     28      & 1C          & FS           & \verb|^\|      &             & File Separator               \\
%     29      & 1D          & GS           & \verb|^]|      &             & Group Separator              \\
%     30      & 1E          & RS           & \verb|^^|      &             & Record Separator             \\
%     31      & 1F          & US           & \verb|^_|      &             & Unit Separator               \\
%     127     & 7F          & DEL          & \verb|^?|      &             & Delete                       \\ \hline
%   \end{tabular}
%   \caption{Caracteres de controle da tabela ASCII.}
%   \label{tab:control}
% \end{table}

\textbf{Identificadores} são sequências de letras maiúsculas ou
minúsculas, dígitos decimais e sublinhados (\texttt{\_}), começando
com uma letra. Letras maiúsculas e minúsculas são distintas em um
identificador.


\begin{comment}
\section{Atividade: front-end do compilador}

\begin{task}[breakable]{Implementação}{}
Implementar o \emph{front-end} do compilador de \lang{}, incluindo:
\begin{enumerate}
  \item analisador léxico, usando o gerador de analisadores léxico
  \emph{JFlex}
  \item analisador sintático, usando o gerador de analisador sintático
  \emph{CUP}
  \item representação dos programas como árvores de sintaxe abstrata,
  usando ações semânticas nas regrras de produção da gramática livre
  de contexto usada pelo \emph{CUP}
\end{enumerate}

A aplicação deverá aceitar o nome do programa fonte a ser compilado
como argumento da linha de comando, realizar as análise léxica e
sintática do mesmo, e exibir a árvore de sintaxe abstrata quando não
houver erros léxicos ou sintáticos.

As seguintes construções são opcionais:
\begin{itemize}
  \item comentários de bloco
  \item strings
  \item expressão de repetição
\end{itemize}
\end{task}

\begin{task}[breakable]{Testes}{}
  Desenvolver um conjunto de programas na linguagem \lang{} que
  permita testar todas as construções da linguagem implementadas, e
  usá-los para testar o seu compilador.
\end{task}
\end{comment}


\section{Atividade: análise léxica}

\begin{task}[breakable]{Implementação}{}
  Implemente um analisador léxico para a linguagem \lang{}.

  A sua aplicação deverá aceitar o nome do arquivo a ser analisado como
  argumento na linha de comando, e exibir a sequência de \emph{tokens}
  obtidas pela análise léxica do arquivo.

  Para cada \texttt{token} o seu analisador léxico deverá informar:
  \begin{itemize}
    \item o tipo do \texttt{token},
    \item o valor semântico do \texttt{token}, quando relevante, e
    \item a posição (número da linha e coluna) em que o \texttt{token}
    aparece no programa fonte.
  \end{itemize}

  Todos os possíveis erros léxicos devem ser reportados corretamente
  pelo seu analisador léxico. Ao reportar um erro, deve-se exibir a
  posição (linha e coluna) em que o erro foi detectado, e uma mensagem
  de diagnóstico.

  A estrutura do projeto está disponível, bastando completar as regras
  de análise léxica.

  Por exemplo, a análise léxica do programa fonte seguinte:
  \begin{pygmented}[lang=text]
{# programa para cálculo do fatorial
   de um número inteiro #}

int fatorial(int n) =
  if n > 0 then
    1                           # caso base
  else
    n * fatorial(n-1)           # caso recursivo

int main() =
  printint(fatorial(7))
  \end{pygmented}
  \noindent produz uma lista de símbolos léxicos similar a:
  \begin{verbatim}
4:1-4:4 INT
4:5-4:13 ID(fatorial)
4:13-4:14 LPAREN
4:14-4:17 INT
4:18-4:19 ID(n)
4:19-4:20 RPAREN
4:21-4:22 EQ
5:3-5:5 IF
5:6-5:7 ID(n)
5:8-5:9 GT
5:10-5:11 LITINT(0)
5:12-5:16 THEN
6:5-6:6 LITINT(1)
7:3-7:7 ELSE
8:5-8:6 ID(n)
8:7-8:8 TIMES
8:9-8:17 ID(fatorial)
8:17-8:18 LPAREN
8:18-8:19 ID(n)
8:19-8:20 MINUS
8:20-8:21 LITINT(1)
8:21-8:22 RPAREN
10:1-10:4 INT
10:5-10:9 ID(main)
10:9-10:10 LPAREN
10:10-10:11 RPAREN
10:12-10:13 EQ
11:3-11:12 ID(print_int)
11:12-11:13 LPAREN
11:13-11:21 ID(fatorial)
11:21-11:22 LPAREN
11:22-11:23 LITINT(7)
11:23-11:24 RPAREN
11:24-11:25 RPAREN
12:1-12:1 EOF
  \end{verbatim}
\end{task}

\begin{task}[breakable]{Testes Automáticos}{}
  Ceritifique-se de que todos os testes implementados no projeto
  sucedam.
\end{task}

\begin{task}[breakable]{Testes}{}
  Teste o seu analisador léxico de \lang{} com os seguintes programas
  fontes:
  \begin{enumerate}
    \item 
\begin{pygmented}[lang=text]
bool f(string s) =
  s = "pedro" || true
\end{pygmented}

    \item 
\begin{pygmented}[lang=text]
int main() =
  let a = 873 in
    let b = a ^ 3 in
      (a + b)/2
\end{pygmented}

    \item 
\begin{pygmented}[lang=text]
int main() =
  # program execution starts here
  let PesoPessoa = 45 in
    print_int(PesoPessoa + 2)
\end{pygmented}

    \item 
\begin{pygmented}[lang=text]
int main() =
  print(-2342,
        56.7,
        "Letra \064.",
        "Escape \k inválida",
        "aspas\"internas\" e \\ barra",
        "ouro \n preto",
        "bom dia)
\end{pygmented}
  \end{enumerate}
\end{task}


\section{Aspectos semânticos}

\end{document}
